\subsection{Experiência 2} \label{exp2_steps}

\textbf{Passo 1}

Começar por fazer as ligações dos cabo

\begin{lstlisting}[language=bash]
TUX23E0  -> Switch Porta 1 (Verificar que luz acende no Switch para saber porta)
TUX24E0  -> Switch Porta 2
TUX22E0  -> Switch Porta 3
\end{lstlisting}

Vista frontal das ligações no switch, lado esquerdo

\begin{table}[h]
\centering
\begin{tabular}{|l|l|l|}
\hline
TUX23E0 (1) & TUX22E0 (3) & empty (5) \\ \hline
TUX24E0 (2) & empty (4)   & empty (6) \\ \hline
\end{tabular}
\caption{\label{tab:table-name}Vista frontal das ligações no switch, lado esquerdo.}
\end{table}

\begin{lstlisting}[language=bash]
tux23:
> ifconfig eth0 up
> ifconfig eth0 172.16.20.1/24
> ifconfig eth0 

tux24:
> ifconfig eth0 up
> ifconfig eth0 172.16.20.254/24
> ifconfig eth0 

tux22:
> ifconfig eth0 up
> ifconfig eth0 172.16.21.1/24
> ifconfig eth0 
\end{lstlisting}


\begin{table}[h]
\centering
\begin{tabular}{|l|l|l|}
\hline
IP            & MAC               & tux/ether  \\ \hline
172.16.21.1   & 00:21:5a:61:2b:72 & tux22 eth0 \\ \hline
172.16.20.1   & 00:21:5a:5a:7d:12 & tux23 eth0 \\ \hline
172.16.20.254 & 00:08:54:50:3f:2c & tux24 eth0 \\ \hline
\end{tabular}
\caption{\label{tab:table-name}Endreços IP, Endereços MAC e interfaces correspondentes.}
\end{table}

\textbf{Passo 2}

Ligar Cabo 

\begin{lstlisting}[language=bash]
TUX23S0  -> T3
T4 -> Switch Console
\end{lstlisting}

Para criar as vlans é necessário modificar o switch, para isso escolhe-se um tux e liga-se a porta série desse tux á porta série da consola do switch. Abre-se o terminal GtKterm, prime-se Enter para verificar que há ligação, depois pode-se seguir os passos abaixo.

Estes passos só precisam de ser realizados uma vez, não é necessário fazer em todos os tux's e não é preciso alterar cabos.

Dar login no switch:
\begin{lstlisting}[language=bash]
>enable
>password: ****** 
\end{lstlisting}

Criar VLAN (vlan20):
\begin{lstlisting}[language=bash]
>configure terminal
>vlan 20
>end
>show vlan id 20
\end{lstlisting}

Adicionar ports: Os valores das ports são os números que acendem ao ligar os cabos ao switch

Liga-se as ports 1 e 2 á VLAN20

Adicionar porta 1 à vlan 20:
\begin{lstlisting}[language=bash]
>configure terminal
>interface fastethernet 0/1 
>switchport mode access
>switchport access vlan 20
>end
>show running-config interface fastethernet 0/1
>show interfaces fastethernet 0/1 switchport
\end{lstlisting}
Adicionar porta 2 à vlan 20:
\begin{lstlisting}[language=bash]
>configure terminal
>interface fastethernet 0/2            
>switchport mode access
>switchport access vlan 20
>end
\end{lstlisting}


\textbf{Passo 3}

Criar outra VLAN (vlan21):
\begin{lstlisting}[language=bash]
>configure terminal
>vlan 21
>end
>show vlan id 21
\end{lstlisting}
Adicionar a porta do tux22, porta 3:
\begin{lstlisting}[language=bash]
>configure terminal
>interface fastethernet 0/3            
>switchport mode access
>switchport access vlan 21
>end
\end{lstlisting}

\textbf{Passo 4}

Antes de fazer a captura no wireshark, verifica-se que os pings que se vão fazer abaixo funcionam. assim que estiver tudo bem, prepara-se os comandos nas consolas e pode-se começar a captura no WireShark.

\textbf{Passo 5}

No tux23, pingar o tux4:
\begin{lstlisting}[language=bash]
ping 172.16.20.254
\end{lstlisting}
Pingar o tux22:
\begin{lstlisting}[language=bash]
ping 172.16.21.1
\end{lstlisting}

\textbf{Passo 6}

Parar a captura e guardar o ficheiro.

\textbf{Passo 7}

Antes de fazer estas capturas prepara-se o ping de baixo, depois segue-se estes passos:

\begin{enumerate}
  \item Em tux23 começar captura em eth0
  \item Trocar para tux24
  \item Em tux24 começar captura em eth0
  \item Trocar para tux22
  \item Em tux22 começar captura em eth0
\end{enumerate}
  
\textbf{Passo 8}

No tux23, pingar em broadcast:
\begin{lstlisting}[language=bash]
ping -b 172.16.20.255
\end{lstlisting}

\textbf{Passo 9}

\begin{enumerate}
  \item Parar a captura em tux23
  \item Parar a captura em tux24
  \item Parar a captura em tux22
  \item Guardar o ficheiro de tux23
  \item Guardar o ficheiro de tux24
  \item Guardar o ficheiro de tux22
\end{enumerate}


\textbf{Passo 10}

Antes de fazer estas capturas prepara-se o ping de baixo, depois segue-se estes passos:

\begin{enumerate}
  \item Em tux23 começar captura em eth0
  \item Trocar para tux24
  \item Em tux24 começar captura em eth0
  \item Trocar para tux22
  \item Em tux22 começar captura em eth0
\end{enumerate}

No tux22, pingar em broadcast:
\begin{lstlisting}[language=bash]
ping -b 172.16.21.255
\end{lstlisting}

\begin{enumerate}
  \item Parar a captura em tux23
  \item Parar a captura em tux24
  \item Parar a captura em tux22
  \item Guardar o ficheiro de tux23
  \item Guardar o ficheiro de tux24
  \item Guardar o ficheiro de tux22
\end{enumerate}