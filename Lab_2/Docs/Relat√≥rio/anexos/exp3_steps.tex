\subsection{Experiência 3} \label{exp3_steps}

\textbf{Passo 1}

Começar por fazer as ligações dos cabos:

\begin{lstlisting}[language=bash]
TUX23E0  -> Switch Porta 1 
TUX22E0  -> Switch Porta 2
TUX24E0  -> Switch Porta 3
TUX24E1  -> Switch Porta 4
\end{lstlisting}

\begin{table}[h]
    \centering
    \begin{tabular}{|l|l|l|}
    \hline
        TUX23E0 (1) & TUX24E0 (3) & empty (5) \\ \hline
        TUX22E0 (2) & TUX24E1 (4) & empty (6) \\ \hline
    \end{tabular}
    \caption{\label{tab:table-name}Vista frontal das ligações no switch, lado esquerdo.}
\end{table}

Configurar IP's:

tux23:
\begin{lstlisting}[language=bash]
> ifconfig eth0 up
> ifconfig eth0 172.16.20.1/24
> ifconfig eth0 
\end{lstlisting}
tux24:
\begin{lstlisting}[language=bash]
> ifconfig eth0 up
> ifconfig eth0 172.16.20.254/24
> ifconfig eth0 

> ifconfig eth1 up
> ifconfig eth1 172.16.21.253/24
> ifconfig eth1 
\end{lstlisting}
tux22:
\begin{lstlisting}[language=bash]
> ifconfig eth0 up
> ifconfig eth0 172.16.21.1/24
> ifconfig eth0 
\end{lstlisting}

\begin{table}[h]
    \centering
    \begin{tabular}{|l|l|l|}
    \hline
        IP & MAC & tux/ether \\ \hline
        172.16.21.1 & 00:21:5a:61:2b:72 & tux22 eth0 \\ \hline
        172.16.20.1 & 00:21:5a:5a:7d:12 & tux23 eth0 \\ \hline
        172.16.20.254 & 00:08:54:50:3f:2c & tux24 eth0 \\ \hline
        172.16.21.253 & 00:22:64:a6:a4:f1 & tux24 eth1 \\ \hline
    \end{tabular}
    \caption{\label{tab:table-name}Endreços IP, Endereços MAC e interfaces correspondentes.}
\end{table}

Configurar VLAN's:

\begin{lstlisting}[language=bash]
TUX23S0  -> T3
T4 -> Switch Console
\end{lstlisting}
Liga-se um Cabo `S0`, TUX23S0 por exemplo, á porta 23 da prateleira de cima (`T3`) e um cabo da porta `T4` a `switch console`.

Esta secção é para ser feita apenas uma vez, num tux á escolha, a partir do terminal GtkTerm, sem necessidade de alterar cabos.

VLAN 0:
\begin{lstlisting}[language=bash]
 - tux23 eth0 -> port 1
 - tux24 eth0 -> port 3
\end{lstlisting}

VLAN 1:
\begin{lstlisting}[language=bash]
 - tux22 eth0 -> port 2
 - tux24 eth1 -> port 4
\end{lstlisting}

Intruções Detalhadas:

Dar login no switch:
\begin{lstlisting}[language=bash]
>enable
>password: ******
\end{lstlisting}

Criar VLAN (vlan20):
\begin{lstlisting}[language=bash]
>configure terminal
>vlan 20
>end
>show vlan id 20
\end{lstlisting}

Adicionar porta 1 a vlan 20:
\begin{lstlisting}[language=bash]
>configure terminal
>interface fastethernet 0/1
>switchport mode access
>switchport access vlan 20
>end
\end{lstlisting}

Adicionar porta 3 a vlan 20:
\begin{lstlisting}[language=bash]
>configure terminal
>interface fastethernet 0/3             
>switchport mode access
>switchport access vlan 20
>end
\end{lstlisting}

Criar VLAN (vlan21):
\begin{lstlisting}[language=bash]
>configure terminal
>vlan 21
>end
>show vlan id 21
\end{lstlisting}

Adicionar porta 2 a vlan 21:
\begin{lstlisting}[language=bash]
>configure terminal
>interface fastethernet 0/2             
>switchport mode access
>switchport access vlan 21
>end
\end{lstlisting}

Adicionar porta 4 a vlan 21:
\begin{lstlisting}[language=bash]
>configure terminal
>interface fastethernet 0/4             
>switchport mode access
>switchport access vlan 21
>end
\end{lstlisting}

No final verificar se está tudo correto com:
\begin{lstlisting}[language=bash]
>show vlan
\end{lstlisting}

Também se pode verificar com:
\begin{lstlisting}[language=bash]
>show running-config interface fastethernet 0/1
>show interfaces fastethernet 0/1 switchport
\end{lstlisting}
testando com números de portas diferentes.

\textbf{Enable IP forwarding and Disable ICMP echo-ignore-broadcast}:

Troca-se para o tux24 e faz-se os seguintes comando no terminal:
\begin{lstlisting}[language=bash]
echo 1 > /proc/sys/net/ipv4/ip_forward
echo 0 > /proc/sys/net/ipv4/icmp_echo_ignore_broadcasts
\end{lstlisting}


\textbf{Passo 2}

Valores de IP e MAC da tabela acima

\textbf{Passo 3}

No tux23 adiciona-se uma rota para dizer que para aceder á vlan com endereços 172.16.21.0/24 usa-se como \textbf{gateway} o IP 172.16.20.254. Este comando é para ser feito na consola:
\begin{lstlisting}[language=bash]
route add -net 172.16.21.0/24 gw 172.16.20.254
\end{lstlisting}
Da mesma forma, no tux22 faz-se:
\begin{lstlisting}[language=bash]
route add -net 172.16.20.0/24 gw 172.16.21.253
\end{lstlisting}

Testa-se pingar o tux22 a partir do tux23 e o oposto para ver se existe conectividade.
\begin{lstlisting}[language=bash]
Em tux23:
ping 172.16.21.1

Em tux22:
ping 172.16.20.1
\end{lstlisting}


\textbf{Passo 4}

Fazer `route -n` em cada 1 dos 3 tuxs para observar as routes.

tux22:
\begin{table}[h]
    \centering
    \begin{tabular}{|l|l|l|l|l|l|l|l|}
    \hline
        Destination & Gateway & Genmask & Flags & Metric & Ref & Use & Iface \\ \hline
        172.16.20.0 & 172.16.21.253 & 255.255.255.0 & UG & 0 & 0 & 0 & eth0 \\ \hline
        172.16.21.0 & 0.0.0.0 & 255.255.255.0 & U & 0 & 0 & 0 & eth0 \\ \hline
    \end{tabular}
    \caption{\label{tab:table-name}Tabela de encaminhamento para o tux22.}
\end{table}

tux23:
\begin{table}[h]
    \centering
    \begin{tabular}{|l|l|l|l|l|l|l|l|}
    \hline
        Destination & Gateway & Genmask & Flags & Metric & Ref & Use & Iface \\ \hline
        172.16.20.0 & 0.0.0.0 & 255.255.255.0 & U & 0 & 0 & 0 & eth0 \\ \hline
        172.16.21.0 & 172.16.21.254 & 255.255.255.0 & UG & 0 & 0 & 0 & eth0 \\ \hline
    \end{tabular}
    \caption{\label{tab:table-name}Tabela de encaminhamento para o tux23.}
\end{table}

tux24:
\begin{table}[h]
    \centering
    \begin{tabular}{|l|l|l|l|l|l|l|l|}
    \hline
        Destination & Gateway & Genmask & Flags & Metric & Ref & Use & Iface \\ \hline
        172.16.20.0 & 0.0.0.0 & 255.255.255.0 & U & 0 & 0 & 0 & eth0 \\ \hline
        172.16.21.0 & 0.0.0.0 & 255.255.255.0 & U & 0 & 0 & 0 & eth1 \\ \hline
    \end{tabular}
    \caption{\label{tab:table-name}Tabela de encaminhamento para o tux24.}
\end{table}

\textbf{Passo 5}

Trocar para o tux23. Ligar o WireShark e começar a capturar pacotes na interface eth0

\textbf{Passo 6}

A partir do tux23:
\begin{enumerate}
  \item pingar a eth0 do tux24 - ping 172.16.20.254
  \item pingar a eth1 do tux24 - ping 172.16.21.253
  \item pingar a eth0 do tux22 - ping 172.16.21.1
\end{enumerate}

Para cada um verificar a conectividade

\textbf{Passo 7}

Parar a captura no tux23 e guardar logs.

\textbf{Passo 8}

\begin{enumerate}
  \item Trocar para o tux24.
  \item Ligar duas instâncias de WireShark. Uma para o eth0 e outra para o eth1. 
  \item Começar a capturar na eth0 e começar a capturar na eth1.
\end{enumerate}

\textbf{Passo 9}

No tux24, apagar a tabela ARP e verificar que está vazia
\begin{lstlisting}[language=bash]
> arp -a (verificar quais os IPs que se podem apagar)
> arp -d 172.16.20.254 (um dos IPs, tem de ser feito com todos os IPs de cima)
> arp -a (tem de retornar nada)
\end{lstlisting}
Fazer o mesmo no tux22 e tux23

\textbf{Passo 10}

No tux23 começar a pingar o tux22 (ping 172.16.21.1).

\textbf{Passo 11}

Trocar para tux24, parar as capturas e guardar os ficheiros.